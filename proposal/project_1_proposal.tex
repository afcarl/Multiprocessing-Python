\documentclass[11pt, oneside]{article}   	% use "amsart" instead of "article" for AMSLaTeX format
\usepackage{geometry}                		% See geometry.pdf to learn the layout options. There are lots.
\geometry{letterpaper}                   		% ... or a4paper or a5paper or ... 
%\geometry{landscape}                		% Activate for rotated page geometry
%\usepackage[parfill]{parskip}    		% Activate to begin paragraphs with an empty line rather than an indent
\usepackage{graphicx}				% Use pdf, png, jpg, or eps§ with pdflatex; use eps in DVI mode
								% TeX will automatically convert eps --> pdf in pdflatex		
\usepackage{amssymb}
\usepackage[colorlinks = true, urlcolor = blue]{hyperref}



%SetFonts

%SetFonts


\title{MSAN 610 \\Project 1 Proposal}
\author{Brendan J. Herger}
%\date{}							% Activate to display a given date or no date

\begin{document}
\maketitle
\section{Proposal}
For my MSAN 610 Project 1 presentation, I would discuss the theory related to and one implementation of \href{https://docs.python.org/2/library/multiprocessing.html#multiprocessing.pool.multiprocessing.Pool.map}{Pool Mapping in Python} (multithreaded mapping using a built in Python module). In particular, I my presentation would aim to inform the audience of the use cases, benefits and detriments of Pool Mapping.

\subsection{Theory}
For the theory section of my presentation, I will briefly discuss how mapping works, as well as a very brief discussion of how multiprocessing works. Additionally, I will mention the speed improvements and overhead detriments associated with using a Pool map. 

Additionally, I will $not$ discuss the need for multiprocessing in lieu of multithreading in Python, nor the Global Interpreter Lock in Python.

\subsection{Implementation}
For the implementation section, I will provide code snippets (to be available digitally on my GitHub), and possibly a live demo which highlights the speed-ups provided by using multiprocessing. The code I will provide will be designed to quickly replace the built in map or list comprehension.


%\subsection{}



\end{document}  